\htmlhr
\chapterAndLabel{Non-Empty Checker for collections, iterators, iterables, and maps}{non-empty-checker}

The Non-Empty Checker warns about operations that depend on whether a
collection, iterator, iterable, or map is non-empty.

To run the Non-Empty Checker, run either of these commands:

\begin{alltt}
  javac -processor nonempty \emph{MyJavaFile}.java
  javac -processor org.checkerframework.checker.nonempty.NonEmptyChecker \emph{MyJavaFile}.java
\end{alltt}

\sectionAndLabel{Non-Empty annotations}{non-empty-annotations}

These qualifiers make up the Non-Empty type system:

\begin{description}

\item[\refqualclass{checker/nonempty/qual}{UnknownNonEmpty}]
  The annotated collection, iterator, iterable, or map may or may not be empty.
  This is the top type; programmers need not explicitly write it.

\item[\refqualclass{checker/nonempty/qual}{NonEmpty}]
  The annotated collection, iterator, iterable, or map is \emph{definitely}
  non-empty.

\item[\refqualclass{checker/nonempty/qual}{PolyNonEmpty}]
  indicates qualifier polymorphism.
  For a description of qualifier polymorphism, see
  Section~\ref{method-qualifier-polymorphism}.

\end{description}

\sectionAndLabel{Annotating your code with \<@NonEmpty>}{annotating-with-non-empty}

The default annotation for collections, iterators, iterables, and maps is
\<@UnknownNonEmpty>.
Refinement to the \<@NonEmpty> type occurs in certain cases, such as after
conditional checks for empty/non-emptiness (see~\ref{type-refinement} for
more details):

\begin{Verbatim}
    public List<String> getSessionIds() { ... }
    ...
    List<String> sessionIds = getSessionIds(); // sessionIds has type @UnknownNonEmpty
    ...
    if (!sessionIds.isEmpty()) {
      List<String> firstId = sessionIds.get(0); // OK, sessionIds has type @NonEmpty
      ...
    }
\end{Verbatim}

Or on the result of a method that returns a non-empty collection:

\begin{Verbatim}
    List<String> countryCodes1; // Has default type @UnknownNonEmpty
    List<String> countryCodes1 = List.of("CA", "US"); // Has type @NonEmpty
\end{Verbatim}

A programmer can manually annotate code in cases where a collection,
iterator, iterable, or map is always known to be non-empty, but that fact is
unable to be inferred by the type system:

\begin{Verbatim}
    // This call always returns a non-empty map; there is always at least one user in the store
    public @NonEmpty Map<UserId, User> getUserMapping() { ... }
    ...
    Map<UserId, User> users = getUserMapping(); // users has type @NonEmpty
\end{Verbatim}

\sectionAndLabel{What the Non-Empty Checker checks}{non-empty-checker-checks}

The Non-Empty Checker ensures that collections, iterators, iterables, or maps
are non-empty at certain points in a program.
If a program type-checks cleanly under the Non-Empty Checker (i.e., no errors
are issued by the checker), then the program is certified with a compile-time
guarantee of the absence of errors rooted in the use of operations that
rely on whether a collection is non-empty.
For example, calling \<next()> on an iterator that is known to be \<@NonEmpty>
should never fail, or, getting the first element of a \<@NonEmpty> list should
not throw an exception.

The Non-Empty Checker does \emph{not} provide guarantees about the fixed
length or size of collections, iterators, iterables, or maps, beyond whether
it has a length of at least 1 (i.e., it is non-empty).
The Index Checker~(See Chapter~\ref{index-checker}) is a checker that analyzes
array bounds and indices and warns about potential
\<IndexOutOfBoundsException>s.

