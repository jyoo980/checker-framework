\htmlhr
\chapterAndLabel{Non-Empty Checker for container classes}{non-empty-checker}

The Non-Empty Checker tracks whether a container is possibly-empty or is
definitely non-empty.  It works on containers such as
\<Collection>s, \<Iterator>s, \<Iterable>s, and \<Map>s.

If the Non-Empty Checker issues no warnings, then your program does not
throw \<NoSuchElementException> as a result of calling methods such as
\<Deque.getFirst()>, \<Deque.removeFirst()>, \<Queue.remove()>, or
\<Queue.element()>.

To run the Non-Empty Checker, run either of these commands:

\begin{alltt}
  javac -processor nonempty \emph{MyJavaFile}.java
  javac -processor org.checkerframework.checker.nonempty.NonEmptyChecker \emph{MyJavaFile}.java
\end{alltt}


\sectionAndLabel{Non-Empty annotations}{non-empty-annotations}

These qualifiers make up the Non-Empty type system:

\begin{description}

\item[\refqualclass{checker/nonempty/qual}{UnknownNonEmpty}]
  The annotated collection, iterator, iterable, or map may or may not be empty.
  This is the top type; programmers need not explicitly write it.

\item[\refqualclass{checker/nonempty/qual}{NonEmpty}]
  The annotated collection, iterator, iterable, or map is \emph{definitely}
  non-empty.

\item[\refqualclass{checker/nonempty/qual}{PolyNonEmpty}]
  indicates qualifier polymorphism.
  For a description of qualifier polymorphism, see
  Section~\ref{method-qualifier-polymorphism}.

\end{description}

\begin{figure}
\includeimage{nonempty-subtyping}{3.75cm}
\caption{The subtyping relationship of the Non-Empty Checker's qualifiers.}
\label{fig-nonempty-hierarchy}
\end{figure}

\subsectionAndLabel{Non-Empty method annotations}{non-empty-method-annotations}

The Non-Empty Checker supports several annotations that specify method
behavior.  These are declaration annotations, not type annotations;  they
apply to the annotated method itself rather than to some particular type.

\begin{description}

\item[\refqualclass{checker/nonempty/qual}{RequiresNonEmpty}]
  indicates a method precondition.  The annotated method expects the
  specified expresssion to be non-empty when the
  method is invoked. \<@RequiresNonEmpty> may be appropriate for
  a field that may not always be non-empty, but the annotated method requires
  the field to be non-empty.

\item[\refqualclass{checker/nonempty/qual}{EnsuresNonEmpty}]
  indicates a method postcondition. The successful return (i.e., a
  non-exceptional return) of the annotated method results in the given
  expression being non-empty. See the Javadoc for examples of its use.

\item[\refqualclass{checker/nonempty/qual}{EnsuresNonEmptyIf}]
  indicates a method postcondition.  With \<@EnsuresNonEmpty>, the given
  expression is non-empty after the method returns normally.  With
  \<@EnsuresNonEmptyIf>, if the annotated
  method returns the given boolean value (true or false), then the given
  expression is non-empty. See the Javadoc for examples of their use.

\end{description}


\sectionAndLabel{Annotating your code with \<@NonEmpty>}{annotating-with-non-empty}

The default annotation for collections, iterators, iterables, and maps is
\<@UnknownNonEmpty>.
Refinement to the \<@NonEmpty> type occurs in certain cases, such as after
conditional checks for empty/non-emptiness (see~\ref{type-refinement} for
more details):

\begin{Verbatim}
    public List<String> getSessionIds() { ... }
    ...
    List<String> sessionIds = getSessionIds(); // sessionIds has type @UnknownNonEmpty
    ...
    if (!sessionIds.isEmpty()) {
      sessionIds.iterator().next(); // OK, sessionIds has type @NonEmpty
      ...
    }
\end{Verbatim}

Or on the result of a method that returns a non-empty collection:

\begin{Verbatim}
    List<String> countryCodes; // Has default type @UnknownNonEmpty
    List<String> countryCodes = List.of("CA", "US"); // Has type @NonEmpty
\end{Verbatim}

A programmer can manually annotate code in cases where a collection,
iterator, iterable, or map is always known to be non-empty, but that fact is
unable to be inferred by the type system:

\begin{Verbatim}
    // This call always returns a non-empty map; there is always at least one user in the store
    public @NonEmpty Map<UserId, User> getUserMapping() { ... }
    ...
    Map<UserId, User> users = getUserMapping(); // users has type @NonEmpty
\end{Verbatim}

\sectionAndLabel{What the Non-Empty Checker checks}{non-empty-checker-checks}

The Non-Empty Checker ensures that collections, iterators, iterables, or maps
are non-empty at certain points in a program.
If a program type-checks cleanly under the Non-Empty Checker (i.e., no errors
are issued by the checker), then the program is certified with a compile-time
guarantee of the absence of errors rooted in the use of operations that
rely on whether a collection is non-empty.
For example, calling \<next()> on an iterator that is known to be \<@NonEmpty>
should never fail, or, getting the first element of a \<@NonEmpty> list should
not throw an exception.

The Non-Empty Checker does \emph{not} provide guarantees about the fixed
length or size of collections, iterators, iterables, or maps, beyond whether
it has a length or size of at least 1 (i.e., it is non-empty).
The Index Checker~(See Chapter~\ref{index-checker}) is a checker that analyzes
array bounds and indices and warns about potential
\<IndexOutOfBoundsException>s.

\sectionAndLabel{Suppressing non-empty warnings}{suppressing-non-empty-warnings}

Like any sound static analysis tool, the Non-Empty Checker may issue a warning
for code that is correct.
It is often best to change your code or annotations in this case.
Alternatively, you may choose to suppress the warning.
This does not change the code, but prevents the warning from being presented to
you.
The Checker Framework supplies several mechanisms to suppress warnings.
See Chapter~\ref{suppressing-warnings} for additional usages.
The \<@SuppressWarnings("nonempty")> annotation is specific to warnings raised
by the Non-Empty Checker:

\begin{Verbatim}
    // This method might return an empty list, depending on the argument
    List<String> getRegionIds(String region) { ... }

    void parseRegions() {
      @SuppressWarnings("nonempty") // A non-empty list is returned when getRegionIds is invoked with argument x
      @NonEmpty List<String> regionIds = getRegionIds(x);
    }
\end{Verbatim}

\subsectionAndLabel{Suppressing warnings with assertions}{suppressing-warnings-with-assertions}

Occasionally, it is inconvenient or verbose to use the \<@SuppressWarnings>
annotation.
For example, Java does not permit annotations such as \<@SuppressWarnings> to
appear on expressions, static initializers, etc.
Here are two ways to suppress a warning in such cases:

\begin{itemize}
\item
  Create a local variable to hold a subexpression, and
  suppress a warning on the local variable declaration.
\item
  Use the \<@AssumeAssertion> string in
  an \<assert> message (see Section~\ref{assumeassertion}).
\end{itemize}

